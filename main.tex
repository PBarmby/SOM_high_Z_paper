\documentclass[useAMS,usenatbib]{mn2e}
\usepackage{amsmath}
\usepackage{hyperref}
\hypersetup{
    colorlinks=True,
    linkcolor=red,
    filecolor=magenta,      
    urlcolor=cyan,
    citecolor=blue,
}
\usepackage{graphicx}
\usepackage{natbib}
\bibliographystyle{mn2e}
\usepackage{times}
\usepackage{float}
%\usepackage{caption}
\usepackage{subcaption}
\usepackage{multirow}
\usepackage{color,soul}
\usepackage{import}
\usepackage[T1]{fontenc}
\usepackage{ae,aecompl}
\usepackage{amssymb}	% Extra maths symbols
\usepackage{multicol}        % Multi-column entries in tables
\usepackage{bm}		% Bold maths symbols, including upright Greek
\usepackage{pdflscape}	% Landscape pages
%\usepackage{booktabs,fixltx2e}
\usepackage[flushleft]{threeparttable}
\usepackage{pgffor}


\newcommand \kpc        {\,{\rm kpc}}
\newcommand \halpha    {H$\alpha $\ }
\newcommand \um    {$\mu$m\ }
\newcommand \mice {$\mu$m}
\newcommand \nprime {N$^\prime$}
\newcommand \boldit {\textbf{\textit{}}}
\newcommand \eqnprime {N^\prime}
\newcommand \Spitzer {{\it Spitzer }}
\newcommand \GALEX {GALEX }
\newcommand \Herschel {{\it Herschel }}
\newcommand{\sii}{S~{\textsc II}\ } 
\newcommand{\oiii}{O~{\textsc III}\ } 
\newcommand{\hi}{H~{\textsc I}\ }

\newcommand \aaj {A\&A}
\newcommand \aarv {A\&ARv}%: Astronomy and Astrophysics Review (the)
\newcommand \aas{A\&AS}%: Astronomy and Astrophysics Supplement Series
\newcommand \afz {Afz}%: Astrofizika
\newcommand \aj {AJ}%: Astronomical Journal (the)
\newcommand \apss {Ap\&SS}%: Astrophysics and Space Science
\newcommand \apj {ApJ}
\newcommand \apjs {ApJS}%: Astrophysical Journal Supplement Series (the)
\newcommand \araa {ARA\&A} %: Annual Review of Astronomy and Astrophysics
\newcommand \asp {ASP Conf. Ser.}%: Astronomy Society of the Pacific Conference Series
\newcommand \azh {Azh}%: Astronomicheskij Zhurnal
\newcommand \baas {BAAS}%: Bulletin of the American Astronomical Society
\newcommand \mem {Mem. RAS}%: Memoirs of the Royal Astronomical Society
\newcommand \mnassa {MNASSA}%: Monthly Notes of the Astronomical Society of Southern Africa
\newcommand \mnras {MNRAS} %: Monthly Notices of the Royal Astronomical Society
%\newcommand {Nature}%(do not abbreviate)
\newcommand \pasj {PASJ}%: Publications of the Astronomical Society of Japan
\newcommand \pasp {PASP}%: Publications of the Astronomical Society of the Pacific
\newcommand \qjras {QJRAS}%: Quarterly Journal of the Royal Astronomical Society
\newcommand \mex {Rev. Mex. Astron. Astrofis.}%: Revista Mexicana de Astronomia y Astrofisica
%\newcommand {Science }%}%(do not abbreviate)
\newcommand \sva {SvA}%: Soviet Astronomy
\newcommand \aap {APP} %:American Academy of Pediatrics
\newcommand \apjl {ApJL} %:The Astrophysical Journal Letters

%\defcitealias{Hossein12}{\citetalias{Hossein12}}

\begin{document}
% TITLE
\defcitealias{Hossein12}{T12}
\defcitealias{Kinney96}{K96}
\defcitealias{Noll09}{N09}

\title[Classifying high-$z$ galaxy spectra]{Classifying galaxy spectra at $0.5<z<1$: an unsupervised approach}
%\author{rahmani.sahar}
\date{\today}
\author[S.~Rahmani, H.~Teimoorinia and P.~Barmby]{S.~Rahmani$^{1,2}$\thanks{E-mail:
srahma49@uwo.ca}, H.~Teimoorinia$^{3}$, P.~Barmby$^{1,2}$\\
$^{1}$Department of Physics $\&$ Astronomy, Western University, London, ON N6A 3K7, Canada\\
$^{2}$Center for Planetary Science \& Exploration, Western University, London, ON N6A 3K7, Canada\\
$^{3}$Department of Physics $\&$ Astronomy, University of Victoria, Finnerty Road, Victoria, British Columbia, V8P 1A1, Canada}
\maketitle

%----------------------------------------------------------------------------------------
%----------------------------------------------------------------------------------------
%----------------------------------------------------------------------------------------
%abstract
%----------------------------------------------------------------------------------------
%----------------------------------------------------------------------------------------
%----------------------------------------------------------------------------------------

\begin{abstract}
The spectral energy distribution (SED) of a galaxy contains information about its physical properties.
Classifying SEDs using templates helps elucidate the nature of a galaxy's energy sources.
However, available templates are limited, which can complicate SED classification.
Most current classification methods cannot classify SEDs outside the range of templates, and will fail in special cases.
In this paper, we introduce a new method of classifying the spectral
energy distributions of high redshift galaxies: the self-organizing map.
Self-organizing maps are a type of artificial neural network, mostly used to identify subgroups in a sample.
We trained an unsupervised self-organizing map networks using a set of
galaxy SED templates covering the wavelength range from far ultraviolet to near infrared.
The trained networks were used to classify the SEDs of a sample of 142 galaxies with $0.5 < z < 1$.
Compared to previous attempts to classify this sample, the
self-organizing map approach was able to better classify the SEDs which had
similarities with more than one template.
The averaged SEDs and physical properties of the galaxies in each neuron of the self-organizing maps are consistent with other results on galaxy evolution.


\end{abstract}
\begin{keywords} 
 galaxies: high red shifts, galaxies: spectral energy distribution, methods: observational, methods: statistical, data mining, methods:data analysis
\end{keywords}
%----------------------------------------------------------------------------------------
%----------------------------------------------------------------------------------------
%----------------------------------------------------------------------------------------
%Intro
%----------------------------------------------------------------------------------------
%----------------------------------------------------------------------------------------
%----------------------------------------------------------------------------------------

\import{sections0.06/}{intro0.06.tex}
%----------------------------------------------------------------------------------------
%----------------------------------------------------------------------------------------
%----------------------------------------------------------------------------------------
%DATA
%----------------------------------------------------------------------------------------
%----------------------------------------------------------------------------------------
%----------------------------------------------------------------------------------------

\import{sections0.06/}{data0.06.tex}

%----------------------------------------------------------------------------------------
%----------------------------------------------------------------------------------------
%----------------------------------------------------------------------------------------
%Method
%----------------------------------------------------------------------------------------
%----------------------------------------------------------------------------------------
%----------------------------------------------------------------------------------------

\import{sections0.06/}{method0.06.tex}

%----------------------------------------------------------------------------------------
%----------------------------------------------------------------------------------------
%----------------------------------------------------------------------------------------
%Results
%----------------------------------------------------------------------------------------
%----------------------------------------------------------------------------------------
%----------------------------------------------------------------------------------------
\import{sections0.06/}{results0.06.tex}
%----------------------------------------------------------------------------------------
%----------------------------------------------------------------------------------------
%----------------------------------------------------------------------------------------
%Summery
%----------------------------------------------------------------------------------------
%----------------------------------------------------------------------------------------
%----------------------------------------------------------------------------------------
\section{SUMMARY AND FUTURE APPLICATIONS}
\label{sec: summary}
Self-organizing maps can be used to classify celestial objects (e.g. stars, quasars, spectra of galaxies, light curves, etc.)
Galaxy spectra can be classified in various networks with different sizes/dimensions based on the information needed from the data. 
If a broad and general classification is required, networks can have one dimension with a few neurons.
If one needs more detailed classifications, a higher number of neurons should be used.
Since self-organizing maps do not include the uncertainty of input parameters, sometimes too much attention to detail can cause problems in classifications. 
When using the SOM method, one should consider whether small differences between objects are physically meaningful when separating two groups from each other.

We used SOMs to classify the template spectra of \citetalias{Kinney96}, made from galaxies with known morphological type, and created networks with different uses.
By varying the size of the networks, we found the relative similarity between the \citetalias{Kinney96} template classes.
A one-dimensional network with 22 neurons was needed in order to
separate all 12 \citetalias{Kinney96} spectra; we concluded that \citetalias{Kinney96} types B and E, and types SB1 and SB2, are very similar to each other.
We also showed that networks generated with the SOM method can be used to easily identify new types of spectra in large surveys.

A sample of 142 high-redshift galaxy SEDs from \citetalias{Hossein12} was classified by the trained networks.
The classification results showed that using SOMs can allow identification of galaxies with SEDs similar to two or more morphological types.
The freedom of having in-between types is one of the main differences between supervised and unsupervised artificial neural networks.
\citetalias{Hossein12} used a supervised training method to train networks with \citetalias{Kinney96} SED templates and classify the same sample of 142 galaxy spectra;
however, they could not classify 37 out of 142 spectra.
The unsupervised method used in this project was able to classify  all 142 spectra in the sample
as belonging to one of the morphological types introduced by \citetalias{Kinney96} or a class in between those types.

The properties of the galaxies in the \citetalias{Hossein12} sample using the new classification were found to be more tightly correlated in mean values of age, specific star formation rate, stellar mass, and far-UV extinction than in previous studies. 
The properties of the galaxies in each group are in good agreement with their morphological types.

\section*{ACKNOWLEDGMENTS}
The authors thank S. Lianou and A. Tammour for their useful comments. 
S.R. and P.B. also acknowledge research support from the Natural Sciences and Engineering Research Council of Canada. 
%----------------------------------------------------------------------------------------
%----------------------------------------------------------------------------------------
%----------------------------------------------------------------------------------------
%biblio
%----------------------------------------------------------------------------------------
%----------------------------------------------------------------------------------------
%----------------------------------------------------------------------------------------
\bibliographystyle{apalike}
\bibliography{ref_mining_h.bib}

\import{sections0.0/}{apps0.0.tex}
\end{document}
