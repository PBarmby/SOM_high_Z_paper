\documentclass[useAMS,usenatbib]{mn2e}
\usepackage{amsmath}
%\usepackage{hyperref}
\usepackage{graphicx}
\usepackage{natbib}
\bibliographystyle{mn2e}
\usepackage{times}
\usepackage{float}
%\usepackage{caption}
\usepackage{subcaption}
\usepackage{multirow}
\usepackage{color,soul}
\usepackage{import}
\usepackage[T1]{fontenc}
\usepackage{ae,aecompl}
\usepackage{amssymb}	% Extra maths symbols
\usepackage{multicol}        % Multi-column entries in tables
\usepackage{bm}		% Bold maths symbols, including upright Greek
\usepackage{pdflscape}	% Landscape pages
%\usepackage{booktabs,fixltx2e}
\usepackage[flushleft]{threeparttable}

\newcommand \kpc        {\,{\rm kpc}}
\newcommand \halpha    {H$\alpha $\ }
\newcommand \um    {$\mu$m\ }
\newcommand \mice {$\mu$m}
\newcommand \nprime {N$^\prime$}
\newcommand \boldit {\textbf{\textit{}}}
\newcommand \eqnprime {N^\prime}
\newcommand \Spitzer {{\it Spitzer }}
\newcommand \GALEX {GALEX }
\newcommand \Herschel {{\it Herschel }}
\newcommand{\sii}{S~{\textsc II}\ } 
\newcommand{\oiii}{O~{\textsc III}\ } 
\newcommand{\hi}{H~{\textsc I}\ }

\newcommand \aaj {A\&A}
\newcommand \aarv {A\&ARv}%: Astronomy and Astrophysics Review (the)
\newcommand \aas{A\&AS}%: Astronomy and Astrophysics Supplement Series
\newcommand \afz {Afz}%: Astrofizika
\newcommand \aj {AJ}%: Astronomical Journal (the)
\newcommand \apss {Ap\&SS}%: Astrophysics and Space Science
\newcommand \apj {ApJ}
\newcommand \apjs {ApJS}%: Astrophysical Journal Supplement Series (the)
\newcommand \araa {ARA\&A} %: Annual Review of Astronomy and Astrophysics
\newcommand \asp {ASP Conf. Ser.}%: Astronomy Society of the Pacific Conference Series
\newcommand \azh {Azh}%: Astronomicheskij Zhurnal
\newcommand \baas {BAAS}%: Bulletin of the American Astronomical Society
\newcommand \mem {Mem. RAS}%: Memoirs of the Royal Astronomical Society
\newcommand \mnassa {MNASSA}%: Monthly Notes of the Astronomical Society of Southern Africa
\newcommand \mnras {MNRAS} %: Monthly Notices of the Royal Astronomical Society
%\newcommand {Nature}%(do not abbreviate)
\newcommand \pasj {PASJ}%: Publications of the Astronomical Society of Japan
\newcommand \pasp {PASP}%: Publications of the Astronomical Society of the Pacific
\newcommand \qjras {QJRAS}%: Quarterly Journal of the Royal Astronomical Society
\newcommand \mex {Rev. Mex. Astron. Astrofis.}%: Revista Mexicana de Astronomia y Astrofisica
%\newcommand {Science }%}%(do not abbreviate)
\newcommand \sva {SvA}%: Soviet Astronomy
\newcommand \aap {APP} %:American Academy of Pediatrics
\newcommand \apjl {ApJL} %:The Astrophysical Journal Letters

%\defcitealias{Hossein12}{T12}

\begin{document}
% TITLE
\defcitealias{Hossein12}{T12}

\title[SOM: classifying high $Z$ galaxies]{Clustering galaxies at 0.5 < $z$ < 1: an unsupervised approach}
\author{rahmani.sahar }
\date{\today}
\author[S.~Rahmani, H.~Teimoorinia and P.~Barmby]{S.~Rahmani$^{1}$\thanks{E-mail:
srahma49@uwo.ca}, H.~Teimoorinia$^{2}$, P.~Barmby$^{1}$\\
$^{1}$Department of Physics $\&$ Astronomy, Western University, London, ON N6A 3K7, Canada\\
$^{2}$Department of Physics $\&$ Astronomy, University of Victoria, Finnerty Road, Victoria, British Columbia, V8P 1A1, Canada}
\maketitle

%----------------------------------------------------------------------------------------
%----------------------------------------------------------------------------------------
%----------------------------------------------------------------------------------------
%abstract
%----------------------------------------------------------------------------------------
%----------------------------------------------------------------------------------------
%----------------------------------------------------------------------------------------

\begin{abstract} %Most of this abstract is just a filler
In this paper we introduce a new method of classifying high red shift galaxies, based on their SED.
We use unsupervised self organizing map to train and validate an artificial neural network which can categorize the spectral energy distribution (SED) of galaxies. 
We utilize the spectral model introduced by Kinney et al. as a training set and a sample of 142 galaxies with 0.5 < $z$ < 1 a validation set.
We obtain physical properties of the sample galaxies measured using {\em CIGALE} code to study and compare the properties of each cluster.
In this paper we introduce a new method of classifying high red shift galaxies, based on their SED.
%%%%%%From this point it is just repeated%%%%%%%%%
We use unsupervised self organizing map to train and validate an artificial neural network which can categorize the spectral energy distribution (SED) of galaxies. 
We utilize the spectral model introduced by Kinney et al. as a training set and a sample of 142 galaxies with 0.5 < $z$ < 1 a validation set.
We obtain physical properties of the sample galaxies measured using CIGALE code to study and compare the properties of each cluster.
We use unsupervised self organizing map to train and validate an artificial neural network which can categorize the spectral energy distribution (SED) of galaxies. 
We utilize the spectral model introduced by Kinney et al. as a training set and a sample of 142 galaxies with 0.5 < $z$ < 1 a validation set.
We obtain physical properties of the sample galaxies measured using CIGALE code to study and compare the properties of each cluster.



\end{abstract}
\begin{keywords} 
 galaxies: star formation, galaxies: stellar content, galaxies: ISM, methods: observational, methods: statistical, data mining, methods:data analysis
\end{keywords}
%----------------------------------------------------------------------------------------
%----------------------------------------------------------------------------------------
%----------------------------------------------------------------------------------------
%Intro
%----------------------------------------------------------------------------------------
%----------------------------------------------------------------------------------------
%----------------------------------------------------------------------------------------

\import{sections0.0/}{intro0.0.tex}
%----------------------------------------------------------------------------------------
%----------------------------------------------------------------------------------------
%----------------------------------------------------------------------------------------
%DATA
%----------------------------------------------------------------------------------------
%----------------------------------------------------------------------------------------
%----------------------------------------------------------------------------------------

\import{sections0.0/}{data0.0.tex}

%----------------------------------------------------------------------------------------
%----------------------------------------------------------------------------------------
%----------------------------------------------------------------------------------------
%Method
%----------------------------------------------------------------------------------------
%----------------------------------------------------------------------------------------
%----------------------------------------------------------------------------------------

\import{sections0.01/}{method0.01.tex}

%----------------------------------------------------------------------------------------
%----------------------------------------------------------------------------------------
%----------------------------------------------------------------------------------------
%Results
%----------------------------------------------------------------------------------------
%----------------------------------------------------------------------------------------
%----------------------------------------------------------------------------------------
\import{sections0.01/}{results0.01.tex}
%----------------------------------------------------------------------------------------
%----------------------------------------------------------------------------------------
%----------------------------------------------------------------------------------------
%Discussion
%----------------------------------------------------------------------------------------
%----------------------------------------------------------------------------------------
%----------------------------------------------------------------------------------------

%\import{sections0.0/}{diss0.0.tex}

%----------------------------------------------------------------------------------------
%----------------------------------------------------------------------------------------
%----------------------------------------------------------------------------------------
%Summery
%----------------------------------------------------------------------------------------
%----------------------------------------------------------------------------------------
%----------------------------------------------------------------------------------------
\section{SUMMARY}
\label{sec: summary}
1) We introduced the self organizing maps to be used to categorizing the SED of galaxies.
\newline
2) Using this method we not only can categorized the galaxies with more precision but also we could recognize the galaxies have similarity to two or more groups of galaxies.
\newline
3) This method can help us to find outliers or new categories of galaxies very easily and fast.

4) We plot the properties of the galaxies using new classification and see the correlation between age and SSFR and stellar mass of the galaxies.


\section*{ACKNOWLEDGMENTS}
S.R. and P.B. acknowledge research support from the Natural Sciences and Engineering Research Council of Canada. This research has made use of the NASA/IPAC Extragalactic Database (NED), which is operated by the Jet Propulsion Laboratory, California Institute of Technology, under contract with the National Aeronautics and Space Administration.
%----------------------------------------------------------------------------------------
%----------------------------------------------------------------------------------------
%----------------------------------------------------------------------------------------
%biblio
%----------------------------------------------------------------------------------------
%----------------------------------------------------------------------------------------
%----------------------------------------------------------------------------------------
\bibliographystyle{mnras}
\bibliography{ref_mining_h.bib}

\import{sections0.0/}{apps0.0.tex}
\end{document}
