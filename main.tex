\documentclass[useAMS,usenatbib]{mn2e}
\usepackage{amsmath}
%\usepackage{hyperref}
\usepackage{graphicx}
\usepackage{natbib}
\bibliographystyle{mn2e}
\usepackage{times}
\usepackage{float}
%\usepackage{caption}
\usepackage{subcaption}
\usepackage{multirow}
\usepackage{color,soul}
\usepackage{import}
\usepackage[T1]{fontenc}
\usepackage{ae,aecompl}
\usepackage{amssymb}	% Extra maths symbols
\usepackage{multicol}        % Multi-column entries in tables
\usepackage{bm}		% Bold maths symbols, including upright Greek
\usepackage{pdflscape}	% Landscape pages
%\usepackage{booktabs,fixltx2e}
\usepackage[flushleft]{threeparttable}

\newcommand \kpc        {\,{\rm kpc}}
\newcommand \halpha    {H$\alpha $\ }
\newcommand \um    {$\mu$m\ }
\newcommand \mice {$\mu$m}
\newcommand \nprime {N$^\prime$}
\newcommand \boldit {\textbf{\textit{}}}
\newcommand \eqnprime {N^\prime}
\newcommand \Spitzer {{\it Spitzer }}
\newcommand \GALEX {GALEX }
\newcommand \Herschel {{\it Herschel }}
\newcommand{\sii}{S~{\textsc II}\ } 
\newcommand{\oiii}{O~{\textsc III}\ } 
\newcommand{\hi}{H~{\textsc I}\ }

\newcommand \aaj {A\&A}
\newcommand \aarv {A\&ARv}%: Astronomy and Astrophysics Review (the)
\newcommand \aas{A\&AS}%: Astronomy and Astrophysics Supplement Series
\newcommand \afz {Afz}%: Astrofizika
\newcommand \aj {AJ}%: Astronomical Journal (the)
\newcommand \apss {Ap\&SS}%: Astrophysics and Space Science
\newcommand \apj {ApJ}
\newcommand \apjs {ApJS}%: Astrophysical Journal Supplement Series (the)
\newcommand \araa {ARA\&A} %: Annual Review of Astronomy and Astrophysics
\newcommand \asp {ASP Conf. Ser.}%: Astronomy Society of the Pacific Conference Series
\newcommand \azh {Azh}%: Astronomicheskij Zhurnal
\newcommand \baas {BAAS}%: Bulletin of the American Astronomical Society
\newcommand \mem {Mem. RAS}%: Memoirs of the Royal Astronomical Society
\newcommand \mnassa {MNASSA}%: Monthly Notes of the Astronomical Society of Southern Africa
\newcommand \mnras {MNRAS} %: Monthly Notices of the Royal Astronomical Society
%\newcommand {Nature}%(do not abbreviate)
\newcommand \pasj {PASJ}%: Publications of the Astronomical Society of Japan
\newcommand \pasp {PASP}%: Publications of the Astronomical Society of the Pacific
\newcommand \qjras {QJRAS}%: Quarterly Journal of the Royal Astronomical Society
\newcommand \mex {Rev. Mex. Astron. Astrofis.}%: Revista Mexicana de Astronomia y Astrofisica
%\newcommand {Science }%}%(do not abbreviate)
\newcommand \sva {SvA}%: Soviet Astronomy
\newcommand \aap {APP} %:American Academy of Pediatrics
\newcommand \apjl {ApJL} %:The Astrophysical Journal Letters

%\defcitealias{Hossein12}{\citealias{Hossein12}}

\begin{document}
% TITLE
\defcitealias{Hossein12}{T12}
\defcitealias{Kinney96}{K96}
\defcitealias{Noll09}{N09}

\title[SOM: classifying high $Z$ galaxies]{Clustering galaxy spectra at $0.5<z<1$: an unsupervised approach} %PB160426: suggest "Clustering galaxy spectra at $0.5<z<1$: an unsupervised approach" -- I think if you say "clustering galaxies" people will think of spatial clustering
\date{\today}
\author[S.~Rahmani, H.~Teimoorinia and P.~Barmby]{S.~Rahmani$^{1,3}$\thanks{E-mail:
srahma49@uwo.ca}, H.~Teimoorinia$^{2}$, P.~Barmby$^{1,3}$\\
$^{1}$Department of Physics $\&$ Astronomy, Western University, London, ON N6A 3K7, Canada\\
$^{2}$Department of Physics $\&$ Astronomy, University of Victoria, Finnerty Road, Victoria, British Columbia, V8P 1A1, Canada\\
$^{3}$Center for Planetary Science \& Exploration, Western University}
\maketitle

%----------------------------------------------------------------------------------------
%----------------------------------------------------------------------------------------
%----------------------------------------------------------------------------------------
%abstract
%----------------------------------------------------------------------------------------
%----------------------------------------------------------------------------------------
%----------------------------------------------------------------------------------------

\begin{abstract}%% (Must be 250 Words)
%PB20160603: are we really classifying SEDs or spectra? Need to think about this.
The spectral energy distribution (SED) of a galaxy contains information about its physical properties.
Classifying SEDs using templates helps elucidate the nature of a galaxy's energy sources.
However, available templates are limited, which can complicate SED classification.
Most current classification methods cannot classify SEDs outside the range of templates, and will fail in special cases.
In this paper, we introduce a new method of classifying the spectral
energy distributions of high redshift galaxies: the self-organizing map.
Self-organizing maps are a type of artificial neural network, mostly used to identify subgroups in a sample.
We trained an unsupervised self-organizing map networks using a set of
galaxy SED templates covering the wavelength range from far ultraviolet to near infrared.
The trained networks were used to cluster the SEDs of a sample of 142 galaxies with $0.5 < z < 1$.
Compared to previous attempts to classify this sample, the
self-organizing map approach was able to better classify the SEDs with
similarities to more than one template.
The averaged SEDs and physical properties of the galaxies in each neuron of the self-organizing maps are consistent with other results on galaxy evolution.
 


%%% what is the problem we are trying to solve in general! why do we want o classify thing in general, and what is wrong in exciting classification scheme

\end{abstract}
\begin{keywords} 
 galaxies: high red shifts, galaxies: spectral energy distribution, methods: observational, methods: statistical, data mining, methods:data analysis
\end{keywords}
%----------------------------------------------------------------------------------------
%----------------------------------------------------------------------------------------
%----------------------------------------------------------------------------------------
%Intro
%----------------------------------------------------------------------------------------
%----------------------------------------------------------------------------------------
%----------------------------------------------------------------------------------------

\import{sections0.04/}{intro0.04.tex}
%----------------------------------------------------------------------------------------
%----------------------------------------------------------------------------------------
%----------------------------------------------------------------------------------------
%DATA
%----------------------------------------------------------------------------------------
%----------------------------------------------------------------------------------------
%----------------------------------------------------------------------------------------

\import{sections0.04/}{data0.04.tex}

%----------------------------------------------------------------------------------------
%----------------------------------------------------------------------------------------
%----------------------------------------------------------------------------------------
%Method
%----------------------------------------------------------------------------------------
%----------------------------------------------------------------------------------------
%----------------------------------------------------------------------------------------

\import{sections0.04/}{method0.04.tex}

%----------------------------------------------------------------------------------------
%----------------------------------------------------------------------------------------
%----------------------------------------------------------------------------------------
%Results
%----------------------------------------------------------------------------------------
%----------------------------------------------------------------------------------------
%----------------------------------------------------------------------------------------
\import{sections0.04/}{results0.04.tex}
%----------------------------------------------------------------------------------------
%----------------------------------------------------------------------------------------
%----------------------------------------------------------------------------------------
%Discussion
%----------------------------------------------------------------------------------------
%----------------------------------------------------------------------------------------
%----------------------------------------------------------------------------------------

%\import{sections0.0/}{diss0.0.tex}

%----------------------------------------------------------------------------------------
%----------------------------------------------------------------------------------------
%----------------------------------------------------------------------------------------
%Summery
%----------------------------------------------------------------------------------------
%----------------------------------------------------------------------------------------
%----------------------------------------------------------------------------------------
\section{SUMMARY AND FUTURE APPLICATIONS}
\label{sec: summary}

% Self organizing maps can be used to classify celestial objects (i.e. stars, quasars, spectra of galaxies, light curves, etc.).
% We showed that we can create various networks with different sizes/dimensions based on information we need from the data. 
% If a broad and general classification is required, networks can have one dimension with few number of neurons.
% While if we need classifications based on differences in more details, we can use a higher number of neurons.
% Since the SOM codes does not work with uncertainty of input parameter, sometimes too much attention to details can cause problems in classifications. 
% Some small differences could easily be the result of atmosphere fluctuation or other instrumental or observational problems. 
% for using SOM method, one should be careful about these small differences and consider in order to separate two groups from each other, how much details do they need to pay attention to.

% We used SOMs to classify SEDs of the galaxies with known morphological type from \citealias{Kinney96}, and created networks with different usages.
% By varying size of networks, we found the relative similarity or dissimilarity between each type of galaxies.
% since we needed a 1D network with 22 neurons to be able to separate all the 12 types of galaxies from \citealias{Kinney96} for the first time, we concluded that types B and E, and types SB1 and SB2 galaxies are really similar to each other.
% We also showed that networks generated with SOM method can be used to easily identified new type of the galaxies or outlier, in large surveys.

% The sample of 142 high red-shift galaxies from \citealias{Hossein12} was used to test the trained networks.
% One of the main criteria to chose a test sample is that the test sample must have fluxes in exactly the same wavelengths of the training sample.
% The test results showed that using SOMs can lead us to identify galaxies with SEDs similar to two or more morphological types but not exactly the same as one of them.
% A freedom of having in between types is one of the main differences between supervised and unsupervised ANNs.
% \citealias{Hossein12}  have used a supervised training method, and trained networks with \citealias{Kinney96} SED template.
% Same as this project, they tried to classify the sample of 142 galaxies using the trained networks.
% However, they could not classify 37 out of 142 galaxies in the sample.
% The unsupervised method that we used in this project, gave us the ability to classify galaxies either as one of the morphological types that have been introduced by \citealias{Kinney96} or between those types. 
% Therefore, we classified all 142 galaxies of our test sample.
% We plotted the properties of the galaxies in test sample using new classification and found a tighter correlations between mean values of age, sSFR, stellar mass, and extinction in FUV band of the galaxies in new classifications than previous studies.
% We also showed he properties of the galaxies in each groups is in good agreement between their morphological types.

Self-organizing maps can be used to classify celestial objects (e.g. stars, quasars, spectra of galaxies, light curves, etc.)
Galaxy spectra can be classified in various networks with different sizes/dimensions based on the information needed from the data. 
If a broad and general classification is required, networks can have one dimension with a few neurons.
If one needs more detailed classifications, a higher number of neurons should be used.
Since self-organizing maps do not include the uncertainty of input parameters, sometimes too much attention to detail can cause problems in classifications. 
%Some small differences could easily be the result of atmosphere fluctuation or other instrumental or observational problems. 
When using the SOM method, one should consider whether small differences between objects are physically meaningful when separating two groups from each other.

We used SOMs to classify the template spectra of \citealias{Kinney96}, made from galaxies with known morphological type, and created networks with different uses.
By varying the size of the networks, we found the relative similarity between the \citealias{Kinney96} template classes.
A one-dimensional network with 22 neurons was needed in order to
separate all 12 \citealias{Kinney96} spectra; we concluded that \citealias{Kinney96} types B and E, and types SB1 and SB2, are very similar to each other.
We also showed that networks generated with the SOM method can be used to easily identify new types of spectra in large surveys.

A sample of 142 high red-shift galaxies from \citealias{Hossein12} was used to test the trained networks.
%One of the main criteria to choose a test sample is that the test sample must have fluxes in exactly the same wavelengths of the training sample.
The test results showed that using SOMs can allow identification of galaxies with SEDs similar to two or more morphological types.% but not exactly the same as one of them.
The freedom of having in-between types is one of the main differences between supervised and unsupervised artificial neural networks.
\citealias{Hossein12} used a supervised training method to train networks with \citealias{Kinney96} SED templates and classify the same sample of 142 galaxy spectra;
however, they could not classify 37 out of 142 spectra.
The unsupervised method used in this project was able to classify  all 142 spectra in the sample
as belonging to one of the morphological types introduced by \citealias{Kinney96} or a class in between those types.

The properties of the galaxies in the test sample using the new
classification were found to more tightly correlate in mean values of age, specific star formation rate, stellar mass, and far-UV extinction than in previous studies. 
The properties of the galaxies in each group are in good agreement with their morphological types.

\section*{ACKNOWLEDGMENTS}
S.R. and P.B. acknowledge research support from the Natural Sciences and Engineering Research Council of Canada. 
%----------------------------------------------------------------------------------------
%----------------------------------------------------------------------------------------
%----------------------------------------------------------------------------------------
%biblio
%----------------------------------------------------------------------------------------
%----------------------------------------------------------------------------------------
%----------------------------------------------------------------------------------------
\bibliographystyle{mnras}
\bibliography{ref_mining_h.bib}

\import{sections0.0/}{apps0.0.tex}
\end{document}
